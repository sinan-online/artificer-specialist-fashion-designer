
\documentclass[twocolumn]{dndbook}

% Compile using UTF-8
\usepackage[utf8]{inputenc}

\usepackage{hyperref}

\hypersetup{
	colorlinks=true, % Whether to color links (a thin box is output around links if this is false)
	%hidelinks, % Hide the default boxes around links
	urlcolor=dirtyorange, % Color for \url and \href links
	linkcolor=black, % Color for \ref/\nameref links
	citecolor=dirtyorange, % Color for reference citations like \cite{}
	hyperindex=true, % Adds links from the page numbers in the index to the relevant page
	linktoc=all, % Link from section names and page numbers in the table of contents
}

\newenvironment{emphasisParagraph}{
	\begin{quote}
	\begin{mdframed}[
		topline=false,
		bottomline=false,
		rightline=false,
		skipabove=\topsep,
		skipbelow=\topsep,
		linecolor=xkcdBottleGreen,
		linewidth=3pt,
	]
	\em
}{
	\end{mdframed}
	\end{quote}
}

\usepackage{titling}
\newcommand{\subtitle}[1]{%
  \posttitle{%
    \par\end{center}
    \begin{center}\large#1\end{center}
    \vskip0.5em}%
}

\usepackage{biblatex}
% \addbibresource{.bib}



\begin{document}

\title{Fashion Designer: A Specialization for Artificers}

\section{Fashion Designer: A Specialization for Artificers}

The fashion designer is able to use fabrics, textiles and other wearable materials to generate social effects.

\subsection{Tool Proficiency}

Starting at the 3rd level, the fashion designer gains proficiency with weaver's tools.\
% TODO: Add reference to the homebrew document.
If they already have this proficiency, they can gain proficiency in leatherworker's tools or cobbler's tools.

\subsection{Artisan's Eye}
The fashion designer has an eye for what people wear and how they carry themselves.
Whenever a check is related to fashion, clothes or accessories, they add a $d6$ to their roll.
This roll increases to $d8$ at the 5th level, $d10$ at the 10th level, $d12$ at the 15th level.
This die stacks with both \emph{Bless} and Bardic Inspiration.\par

For example, the fashion designer can add this die to Int (Investigation) rolls to find out details of a character,
or Cha (Persuasion) rolls to show they use their garments and how they present themselves to get what they want out of social situations.\par

\subsection{Fashion Designer Spells}

Starting at 3rd level, the fashion designer alwats have certau soells prepared after you reach particular levels in this class, shown in the Fashion Designer Spells table.
These spells count as artificer spells for you, but they do ont count against the number of artificer spells your prepare.

\begin{DndTable}[header=Fashion Deisgner SPells]{cl}
    Artificer Level     & Spell \\
    3rd                 &       \\
    5th                 &       \\
    9th                 &       \\
    13th                &       \\
    17th                &       \\
\end{DndTable}

\subsection{Existing Infusions as Fashion Designs}

Beginning at the 3rd level, the fashion designer has access to the following ``designs''.
Designs work as artificer infusions.
At the 3rd level, the designer can swap their known infusions for these designs.
Every time the fashion designer gains a level, they can swap their knowns designs or infusions for other designs.
The total number of designs that they know is determined by the ``infusions known'' table.\par

The following infusions from ``Eberron: Rising from the Last War'' are modified as described below:
\subsubsection{Resistant Armor}
As a fashion design, this infusion can also be used on clothes, not just armor.

\subsubsection{Enhanced Defense}
As a fashion design, this infusion can also be used on clothes, not just armor.

\subsubsection{Repulsion Shield}
As a fashion design, this infusion can also be used on clothes, not just armor.


\subsubsection{Designer Homonculus Servant}
Beginning at the 6th level, the fashion designer has the ability to create a ``homonculus servant''.
However, the homonculus servant is actually an article of clothing or garment, accessory, or even armor.
Articles of clothing can be combined to create the homonculus:
For example, a scarf, a shirt, a cloak, and a pair of trousers or a pair of gloves can each be a homonculus,
but they can all be combined to create one.

The ``designer homonculi'' are different from the regular homonculi in the following ways.
Otherwise, they use the same stat blocks.

\paragraph{Construct Immunities} Blinded, Charmed, Deafened, Frightened, Paralyzed, Petrified, Poisoned

\paragraph{Cannot Attack Unless Worn}
The designer homonculus cannot attack, but can still deliver a touch range spell.
The exception is that shirts other garments worn around the chest can execute a smother attack with advantage \emph{if they are worn}.
On a hit, the creature wearing the garment is grappled (escape DC is the fashion designer's spell save DC).
Until this grapple ends, the target is restrained and at risk of suffocating, but they take no damage.
The attack bonus is +4.
Necklaces and scarves can also execute a smother attack, and the target will still be at risk of suffocation on a missed roll, but they will not be grappled.
Taking off the garment requires a second roll against the same DC, otherwise the designer homonculus can attack again.

\paragraph{Speed}
Unless the ``designer homonculus'' includes a pair of trousers, their speed is 10ft on the ground - they still have 30 ft fly speed.

\paragraph{AC}
Their AC is eithe 12 because of the +2 dexterity bonus, or if a piece of armor, whatever the armor has.

\paragraph{False Appearance}
Unlike the usual homonculus, they have the trait ``false appearance'':
While the mimic remains motionless, it is indistinguishable from an ordinary object.

\paragraph{Damage Transfer}
If worn by a creature, the designer homonculus takes only half the damage dealt to it (rounded down), and that creature takes the other half.

\paragraph{Antimagic Susceptibility}
If targeted by dispel magic, the homonculus must succeed on a Constitution saving throw against the caster's spell save DC or fall unconscious for 1 minute.


\subsection{Fashion Designs (New Infusions)}



\end{document}