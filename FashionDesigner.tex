\documentclass[twocolumn]{dndbook}

% Compile using UTF-8
\usepackage[utf8]{inputenc}

\usepackage{hyperref}

\hypersetup{
	colorlinks=true, % Whether to color links (a thin box is output around links if this is false)
	%hidelinks, % Hide the default boxes around links
	urlcolor=dirtyorange, % Color for \url and \href links
	linkcolor=black, % Color for \ref/\nameref links
	citecolor=dirtyorange, % Color for reference citations like \cite{}
	hyperindex=true, % Adds links from the page numbers in the index to the relevant page
	linktoc=all, % Link from section names and page numbers in the table of contents
}

\newenvironment{emphasisParagraph}{
	\begin{quote}
	\begin{mdframed}[
		topline=false,
		bottomline=false,
		rightline=false,
		skipabove=\topsep,
		skipbelow=\topsep,
		linecolor=xkcdBottleGreen,
		linewidth=3pt,
	]
	\em
}{
	\end{mdframed}
	\end{quote}
}

\usepackage{titling}
\newcommand{\subtitle}[1]{%
  \posttitle{%
    \par\end{center}
    \begin{center}\large#1\end{center}
    \vskip0.5em}%
}

\usepackage{biblatex}
% \addbibresource{.bib}



\begin{document}

\title{Fashion Designer: A Specialization for Artificers}

\subsection{Fashion Designer: A Specialization for Artificers}

The fashion designer is able to use fabrics, textiles and other wearable materials to generate social effects.

\subsubsection{Tool Proficiency}

Starting at the 3rd level, the fashion designer gains proficiency with weaver's tools.\
% TODO: Add reference to the homebrew document.
If they already have this proficiency, they can gain proficiency in leatherworker's tools or cobbler's tools.

\subsubsection{Artisan's Eye}
The fashion designer has an eye for what people wear and how they carry themselves.
Whenever a check is related to fashion, clothes or accessories, they add a $d6$ to their roll.
This roll increases to $d8$ at the 5th level, $d10$ at the 10th level, $d12$ at the 15th level.
This die stacks with both \emph{Bless} and the bardic inspiration.\par

For example, the fashion designer can add this die to Int (Investigation) rolls to find out details of a character,
or Cha (Persuasion) rolls to show they use their garments and how they present themselves to get what they want out of social situations.\par

\subsubsection{Fashion Designer Spells}

% TODO: Add spells

\subsubsection{Existing Infusions as Fashion Designs}

Beginning at the 3rd level, the fashion designer has access to the following ``designs''.
Designs work as artificer infusions.
At the 3rd level, the designer can swap their known infusions for these designs.
Every time the fashion designer gains a level, they can swap their knowns designs or infusions for other designs.
The total number of designs that they know is determined by the ``infusions known'' table.\par

For gameplay purposes, the following infusions from ``Eberron: Rising from the Last War'' count as designs and the fashion designer can choose them:
\begin{itemize}
    \item Boots of the Winding Path
\end{itemize}

The following infusions from ``Eberron: Rising from the Last War'' are modified as described below:
\paragraph{Resistant Armor}
As an fashion design, this infusion can also be used on clothes, not just armor.


\paragraph{Homonculus Servant}
Beginning at the 6th level, the fashion designer has the ability to create a ``homonculus servant''.
However, the homonculus servant is actually an article of clothing or garment, accessory, or even armor.
Unlike the usual homonculus, it has trait ``false appearance'':
While the mimic remains motionless, it is indistinguishable from an ordinary object.

\subsubsection{Fashion Designs (New Infusions`')}

\end{document}